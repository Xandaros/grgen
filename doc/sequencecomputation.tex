%TODO: durchgehen und alles was hier exklusiv ist in die computation-kapitel aufnehmen

\chapter{Sequence Computations}\indexmain{sequence computations}

In this chapter we'll have a look at sequence computations, which are not concerned with directly controlling rules, but with computing values or causing side effects.

%%%%%%%%%%%%%%%%%%%%%%%%%%%%%%%%%%%%%%%%%%%%%%%%%%%%%%%%%%%%%%%%%%%%%%%%%%%%%%%%%%%%%%%%%%%%%%%%
\section{Sequence Computation} \label{sec:seqcomp}

\begin{rail} 
  RewriteComputationUsage: (percent)? lbrace CompoundComputation rbrace; 
\end{rail}\ixnterm{RewriteFactor}

The non-computational constructs introduced before are used for executing rules, to determine which rule to execute next depending on success and failure of the previous rule applications, and where to apply it next by transmitting atomic variables of node or edge type in between the rules.
Sequence computations in contrast are used for manipulating container variables, evaluating computational expressions, or for causing side effects like output or element markings.
The computation will return a boolean value by comparing the return value of the compound computation to the default value of the corresponding type, and returning false if equal, or true if unequal; a computation without a return value always returns true.
So just using a boolean variable as computation returns the value of the variable.
A prepended \texttt{\%} attaches a \indexed{break point} to the computation.

\begin{rail} 
  CompoundComputation: Computation ((';' Computation)*); 
\end{rail}

A compound computation consists of a computation followed by an optional list of computations separated by semicolons.
The computations are executed from left to right;
the value of the compound computation is the value of the last computation.
So you must give an expression there in order to return a value, 
whereas it is pointless to specify an expression before.

\begin{rail} 
  Computation:
     VariableDeclaration |
     Assignment |
     MethodCall |
     ProcedureCall |
     SequenceExpression
  ;
	Assignment:	AssignmentTarget '=' (SequenceExpression | Assignment); 
\end{rail}\ixnterm{Computation}

A variable declaration declares a local variable in the same way as in the sequences.
An assignment assigns the value of a sequence expression to an assignment target.
It may be chained; such an assignment chain is executed from right to left, assigning the rightmost value to all the assignment targets given.
The form of expressions and assignment targets will be specified below.

\begin{rail} 
	MethodCall: (Variable | GraphElement '.' Attribute) (SingleMethodCall +);
	SingleMethodCall: '.' MethodName '(' Arguments ')';
	ProcedureCall: ('(' Variables ')' '=')? ProcedureName '(' Arguments ')';
	Arguments: (SequenceExpression * ',');
	Varibales: (Variable + ',');
\end{rail}\label{recstmt}\indexmain{record}\indexmain{emit}

A method call executes a (predefined) method on a variable, passing further arguments.
It may be chained; such a method call chain is executed from left to right. 
This is possible with storage changing methods which return the variable again, better: which return the then altered variable. 

A procedure call executes a (built-in or user defined) procedure, passing further arguments.
In addition to the graph type based functions and procedures which will be explained in more detail further below,
\texttt{emit}, \texttt{record}, and \texttt{export} procedure calls can be given here: the emit procedure writes a double quoted string or the value of a variable to the emit target (stdout as default, or a file specified with the shell command \texttt{redirect emit}).
The record procedure writes a double quoted string or the value of a variable to the currently ongoing recordings (see \ref{recordnreplay}). This feature allows to mark states reached during the transformation process in order to replay only interesting parts of an recording. It is recommended to write only comment label lines, i.e. \verb/"#"/, some label, and \verb/"\n"/.
The export procedure exports the current graph to the path specified if called with one argument, or it exports the subgraph specified as first argument to the path specified as second argument.
It behaves like the export command from the GrShell, see \ref{outputcmds}.
Having it available in the sequences allows for programmed exporting, and exporting of parts of the graph, with the subgraph containment just computed.

Furthermore, a function \texttt{canonize(g:graph):string} is available,
which is intended to provide a canonical string representation for any graph, but currently does not work for all graphs.
The function currently uses the SMILES\cite{SMILES} method of producing an equitable partition of graph nodes, not a canonical order; while not offering full fledged graph canonization this algorithm is sufficient for many purposes. 
It allows to reduce graph comparisons to string comparisons, at the price of computing the Weininger algorithm for equitable partitions of nodes.

Besides those predefined procedures (and functions), you may call user procedures (and functions), defined in the rules file; cf. \ref{sub:compdef}.

Finally, an expression (without side effects) can be evaluated, this allows to return a (boolean) value from a computation.

\begin{rail}
  AssignmentTarget: 
    Variable (':' Type)? |
    'yield' Variable |
    GraphElement '.' Attribute |
    Variable '[' SequenceExpression ']' |
    GraphElement '.' Attribute '[' SequenceExpression ']' |
    GraphElement '.' 'visited' '[' SequenceExpression ']'
;
\end{rail}\ixnterm{AssignmentTarget}\ixkeyw{visited}\ixkeyw{yield}

Possible targets of assignments are the variables and def-variables to be yielded to, as in the simple assignments of the sequences. 
A \texttt{yield} assignment writes the rhs variable value to the lhs variable which must be declared as a  def-to-be-yielded-to variable (\texttt{def}-prefix) in the pattern containing the \texttt{exec} statement.
Yielding is only possible from compiled sequences, it always succeeds.
Further on, the attributes of graph elements may be written to, the values at given positions of array or deque or map variables may be written to, and the visited status of graph elements may be changed.


%%%%%%%%%%%%%%%%%%%%%%%%%%%%%%%%%%%%%%%%%%%%%%%%%%%%%%%%%%%%%%%%%%%%%%%%%%%%%%%%%%%%%%%%%%%%%%%%
\section{Sequence Expression} \label{sec:seqexpr}

\begin{rail}
  SequenceExpression:  
    ConditionalSequenceExpression |
    BooleanSequenceExpression |
    RelationalSequenceExpression |
    ArtihemticSequenceExpression |
    PrimarySequenceExpression;
\end{rail}\ixnterm{SequenceExpression}

Sequence expressions are a subset of the expressions introduced in \ref{sub:expr}, containing the boolean and comparison operators, but only plus and minus as further operators.

\begin{rail}
  ConditionalSequenceExpression: 
    BooleanSequenceExpression '?' SequenceExpression ':' SequenceExpression;
\end{rail}\ixnterm{ConditionalSequenceExpression}

The conditional operator has lowest priority, if the condition evaluates to true the first expression is evaluated and returned, otherwise the second.

\begin{rail}
  BooleanSequenceExpression: 
    SequenceExpression (ampersand | xorhat | '|' | doubleampersand | '||') SequenceExpression |
    '!' SequenceExpression;
\end{rail}\ixnterm{BooleanSequenceExpression}

The boolean operators have the same semantics and same priority as in \ref{sub:expr}.

\begin{rail}
  RelationalSequenceExpression: 
    SequenceExpression ('==' | '!=' | '<' | '<='| '>' | '>=' | 'in' | titilde) SequenceExpression;
\end{rail}\ixnterm{RelationalSequenceExpression}

The equality operators work for every type and return whether the values to compare are equal or unequal.
The relational operators work as specified in \ref{sub:expr} for numerical types, in \ref{cha:container} for container types, and \ref{cha:graph} for graph type.

\begin{rail}
  ArithmeticSequenceExpression:
    SequenceExpression ('+' | '-') SequenceExpression;
\end{rail}\ixnterm{ArithmeticSequenceExpression}

The arithmetic operator plus is used to denote addition of numerical values or string concatenation,
the arithmetic operator minus is used to denote subtraction of numerical values.
No further arithmetic operators are available in the sequence expressions;
neither are the string methods supported.

\begin{rail}
  PrimarySequenceExpression:
    BasicSequenceExpression |
    SpecialSequenceExpression;
\end{rail}\ixnterm{PrimarySequenceExpression}

The atoms of the sequence expressions are the basic and the special sequence expressions.

\begin{rail}
  BasicSequenceExpression:
    'def' '(' (Variable+',') ')' |
	  railat '(' NameString ')' |
 	  GraphElement '.' Attribute |
	  Variable | 
    Literal
  ;
\end{rail}\ixnterm{BasicSequenceExpression}\ixkeyw{def}

The basic sequence expressions are the foundational value sources.
A \texttt{def} term is successful iff all the variables are defined (not null).
The at operator allows to access a graph element by its \indexed{persistent name}.
The attribute access clause returns the attribute value of the given graph element.
The variable and literal basic expressions are the same as in the \emph{SimpleOrInteractiveExpression}, cf. \ref{sec:simplevarhandling};
this means esp. that a Variable may denote a graph global variable if prefixed with a double colon, here as well as in the AssignmentTarget.

\begin{rail}
  SpecialSequenceExpression:
    Variable '[' SequenceExpression ']' |
    GraphElement '.' Attribute '[' SequenceExpression ']' |
    GraphElement '.' 'visited' '[' SequenceExpression ']' |
    MethodCall |
    ProcedureCall;
  ;
\end{rail}\ixnterm{SpecialSequenceExpression}\ixkeyw{visited}

The special sequence expressions are used for storage and visited flag handling, for random value queries, and for graph and subgraph handling.

The storage and visited flag oriented ones are used to check whether a value is marked, to access a storage, or to call a method on a storage (note: here it is not possible to build method call chains). 
They were explained in the chapters \ref{cha:container} and \ref{cha:graph}.

The random value function \texttt{random} behaves like the random function from the expressions, see \ref{sec:primexpr};
i.e. if noted down with an integer as argument it returns a random integer in between 0 and that upper bound, exclusive; if given without an argument it returns a random double in between 0.0 and 1.0, exclusive.


%%%%%%%%%%%%%%%%%%%%%%%%%%%%%%%%%%%%%%%%%%%%%%%%%%%%%%%%%%%%%%%%%%%%%%%%%%%%%%%%%%%%%%%%%%%%%%%%
\section{Graph and Subgraph Based Queries and Updates}\label{sec:queryupdate}\label{sec:visited}

The graph and subgraph oriented parts of the sequence expressions are built from four groups,
the procedures for basic graph manipulation, the functions for querying the graph structure, the functions and procedures of the subgraph operations, and the visited flag query, assignment, and procedures.

The first group is built from basic graph manipulation operators, as defined in \ref{procstab} and described in \ref{sub:procedures}.
Elements may be added, removed, or retyped, and nodes may be merged or edges redirected.
Not available are numerical functions, they are only offered by the computations of the rules. 

The second group is built from the operators querying primarily the connectedness of graph elements,
as defined in \ref{sub:functions}.
You may ask for one for the nodes or edges of a type.
You may query for the other for the source or target or opposite node of an edge.
Furthermore, you may query for adjacent nodes and incident edges,
maybe even transitively for the reachability.
Furthermore you may ask with a predicate whether nodes or edges are adjacent or incident to other nodes or edges, maybe even transitively for reachability.

The third group is defined by functions and procedures that operate on (sub-)graphs, as defined in \ref{sub:functions} and \ref{sub:procedures}.
They are especially useful in state space enumeration, cf. \ref{sec:statespaceenum}.
You may import, clone, or compute induced subgraphs.
You may export a subgraph or insert a subgraph into the hostgraph.

The fourth group are the visited flags related operations,
as described in chapter \ref{sub:visitedaccess}.
Available is an expression for reading a visited flag, an assignment for writing a visited flag, and procedures for managing the visited flags as defined in \ref{procstab}.

In the sequences only the sequence expressions are available to compute the parameters for the functions and procedures.


%%%%%%%%%%%%%%%%%%%%%%%%%%%%%%%%%%%%%%%%%%%%%%%%%%%%%%%%%%%%%%%%%%%%%%%%%%%%%%%%%%%%%%%%%%%%%%%%
\section{Storage Handling in the Sequences}\label{sec:storages}\indexmain{storage}
Storages are variables of container (set/map/array/deque) type (cf. \ref{sec:builtingenerictypes}) storing nodes or edges.
They are primarily used in the sequences, from where they are handed in to the rules via \texttt{ref} parameters (but additionally container attributes in graph elements may be used as storages,
esp. for doing data flow analyses, cf. \ref{subsub:flow}).
They allow to decouple processing phases: the first run collects all graph elements relevant for the second run which consists of a sequence executed for each graph element in the container.
The splitting of transformations into passes mediated by container valued global variables allows for subgraph copying without model pollution, cf. \ref{subsub:copystructure}; please have a look at \ref{sub:mergesplit}, \ref{subsub:copystructure} and \ref{subsub:flow} regarding a discussion on when to use which transformation combinators and for storage examples.
They were already defined and described in \ref{cha:container}.
Here we only give some refinements and explanations of the semantics.

The methods \texttt{add,rem,clear} are available for all storages and allow to add elements to the container, remove elements from the container, or clear the container.
Their return value is the changed container, thus they allow to chain method calls on the container.

The methods \texttt{size,empty,peek} in contrast return the size of the container, whether the container is empty, or a certain element from the container and thus can't be chained.
The indexed query \texttt{v=m[k]} is available on map and array and deque types and returns the element at the specified index,
the indexed assignment \texttt{a[i]=v} overwrites the element at the specified index.
Further available are the sequence expression operator \texttt{in} for membership query.

The infix operators, the methods specific to a certain container type (i.e. not available for all container types), and the change assignments in contrast are \emph{not} available in the sequence computations, they are \emph{only} supported by the computations in the rule language .

You may iterate in the sequences with a for loop over the elements contained in a storage.

\begin{rail}
  RewriteFactor:
    'for' lbrace (Var 'in' SetVar | Var '->' Var 'in' MapVar | Var 'in' ArrayVar | Var '->' Var 'in' ArrayVar | Var 'in' DequeVar | Var '->' Var 'in' DequeVar) ';' RewriteSequence rbrace
    ;
\end{rail}\ixkeyw{in}\ixkeyw{for}\ixnterm{RewriteFactor}\label{forstorage}

The \texttt{for} command iterates over all elements in the set or array or deque, or all key-value pairs in the map or array or deque, and executes for each element / key-value pair the nested graph rewrite sequence; it completes successfully iff all sequences were executed successfully (an empty container causes immediate successful completion); the key in the key-value pair iteration of an array or deque is the integer typed index. (See \ref{forgraphelem} for another version of the \texttt{for} command.)

\begin{example}
The following XGRS is a typical storage usage.
First an empty set \texttt{x} is created, which gets populated by an rule \texttt{t} executed iteratedly, returning a node which is written to the set.
Then another rule is executed iterated for every member of the set doing the main work, and finally the set gets cleared to prevent memory leaks or later mistakes.
If the graph should stay untouched during set filling you may need \texttt{visited} flags to prevent endless looping.
\verb#x=set<Node>{} ;> ( (v)=t() && {x.add(v)} )+ && for{v in x; r(v)} <; {x.clear()}#
Handing in the storage to the rule, and using the set \texttt{add} method as introduced down below in \ref{sct:imperative} within the rule to fill the storage, allows to shorten the sequence to:\\
\verb#x=set<Node>{} ;> ( t(x) )+ && for{v in x; r(v)} <; {x.clear()}#\\
The for loop could be replaced by employing the storage access in the rule construct, cf. \ref{sub:storageaccess}; this would be especially beneficial if the rule \texttt{r} inside the for loop would have to change the storage \texttt{x}, which would corrupt the iteration/enumeration variable.
\end{example}

\begin{warning}
The container over which the for loop iterates must stay untouched during iteration.
\end{warning}

%\begin{example}
%\begin{grgen}
%::x=set<NodeTypeA>{}
%y:map<Node,Edge> = map<Node,Edge>{u->v}
%\end{grgen}
%The first line declares or references a global variable \texttt{x} (without static type) and assigns the newly created, empty set of type \texttt{set<NodeTypeA>} to it as value.
%The second line declares a variable \texttt{y} of type \texttt{map<Node,Edge>} and assigns the newly created map of the same type containing a key value pair build from u mapping to v, where u is assumed to be a variable of node type and v to be a variable of edge type.
%\end{example}

%%%%%%%%%%%%%%%%%%%%%%%%%%%%%%%%%%%%%%%%%%%%%%%%%%%%%%%%%%%%%%%%%%%%%%%%%%%%%%%%%%%%%%%%%%%%%%%%
\section{Quick Reference Table}

Table~\ref{comptab} lists most of the operations of the graph rewrite computations at a glance.

 %\makeatletter
\begin{table}[htbp]
\begin{minipage}{\linewidth} \renewcommand{\footnoterule}{} 
\begin{tabularx}{\linewidth}{|lX|}
\hline
\texttt{c;d}	& Computes c then d; the value of the computation is d\\
\texttt{t=e}	& Simple assignment of an expression value to an assignment target\\
\texttt{t=e=f}	& Chained assignment \\
\texttt{v.m(e)}	& Simple method call, with m e.g. being storage add \\
\texttt{v.m(e).m(e)}	& Chained method call\\
\hline
\texttt{e ? f : g}	& Returns f if e evaluates to true, otherwise g \\
\texttt{e op f}	& For \texttt{op} being one of the boolean operators \texttt{||,|,\&,\&\&,\^\ } \\
\texttt{e op f}	& For \texttt{op} being one of comparison operators \texttt{==,!=,<,<=,>,>=,in} \\
\texttt{e + f}	& Numerical addition or string concatenation \\
\texttt{e - f}	& Numerical subtraction \\
\hline
\texttt{v} & Variable. Assignment target or expression.\\
\texttt{v.name} & Attribute of graph element. Assignment target or expression.\\
\texttt{@(name)} & Return graph element of given name.\\
\texttt{def(\emph{Parameters})} & Check if all the variables are defined.\\
\texttt{random(upperBound)} & Returns random number from [0;upper bound[, if upper bound is missing from [0.0;1.0[.\\
\hline
\texttt{function(...)}	& One of the functions for graph querying defined in \ref{funcstab} and explained in \ref{cha:graph}. The numerical functions are \emph{not} available.\\
\hline
\texttt{procedure(...)}	& One of the procedures for graph manipulation defined in \ref{procstab} and explained in \ref{cha:graph}.\\
\hline
\texttt{u=set<Node>\{\}}	& Example for container constructor, creates storage set and assigns to \texttt{u}.\\
\texttt{u[e]}	& Target value of \texttt{e} in \texttt{u}. Fails if \texttt{!(e in u)}. Assignment target or expression.\\
\texttt{storage.method(...)}	& One of the methods \texttt{add,rem,clear,size,empty,peek} defined in \ref{cha:container}. The infix operators, methods specific to a certain container type, and change assignments are \emph{not} available.\\
\hline
\end{tabularx}
\end{minipage}\\
\\ 
{\small Let \texttt{c} and \texttt{d} be computations, \texttt{t} be an assignment target, \texttt{e}, \texttt{f}, \texttt{g} be expressions, \texttt{u}, \texttt{v}, \texttt{w} be variable identifiers }
\caption{Sequence computations at a glance}
\label{comptab}
\end{table}
%\makeatother
 
% todo: beispiele im text bringen
