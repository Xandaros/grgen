%TODO: durchgehen und alles was hier exklusiv ist in die computation-kapitel aufnehmen

\chapter{Sequence Computations}\indexmain{sequence computations}\label{seqcomp}

In this chapter we'll take a look at sequence computations, which are not concerned with directly controlling rules, but with computing values or causing side effects, that are then used to control rules.

\begin{example}
Sequence computations are typically employed when storages have to be maintained:
\verb#now:set<Node>=set<Node>{};>next:set<Node>=set<Node>{};>initializeNow(now) ;>#
\verb#( processNowFillNext(now, next) ;> { now.clear(); tmp:set<Node>=now; now=next;#
\verb#next=tmp; {!now.empty()} } )*#
--- that example sequence is used to implement a wavefront as rule control strategy.

A set \texttt{now} of current nodes is processed, filling a set of output nodes to be processed \texttt{next}, which are then used in the following iteration step as input nodes, until all of parts of the graph reachable from the initial nodes have been passed, yielding an empty set.
The sequence computation for switching in between the sets is given enclosed in braces in the sequence, 
the sequence expression to determine whether the wavefront came to a halt is given at the end of the computation,
enclosed a level deeper in braces.
As long as the set is still filled the final expression, thus the computation, returns \texttt{true}, continuing with the loop -- when it gets empty, the expression yields \texttt{false}, henceforth terminating the loop and the wavefront algorithm.

Sequence computations are made available in the sequences for tasks that benefit from state changes and comparisons being embedded directly in the controlling sequence, esp. for variable initializations and loop control.
But as sequences are existing to control rules and not to do the bulk of the computational work,
is only a small subset of the rule computations language available in the sequence compuations. 
\end{example}


%%%%%%%%%%%%%%%%%%%%%%%%%%%%%%%%%%%%%%%%%%%%%%%%%%%%%%%%%%%%%%%%%%%%%%%%%%%%%%%%%%%%%%%%%%%%%%%%
\section{Sequence Statements} \label{sec:seqcomp}

\begin{rail} 
  RewriteComputationUsage: (percent)? lbrace CompoundComputation rbrace; 
\end{rail}\ixnterm{RewriteFactor}

The non-computational constructs introduced before are used for executing rules, to determine which rule to execute next depending on success and failure of the previous rule applications, and where to apply it next by transmitting atomic variables of node or edge type in between the rules.
Sequence computations in contrast are used for manipulating container variables, evaluating computational expressions, or for causing side effects like output or element markings.
A computation returns always true, with exception of an expression used as a computation (explained below).
A prepended \texttt{\%} attaches a \indexed{break point} to the computation.

\begin{rail} 
  CompoundComputation: Computation ((';' Computation)*); 
\end{rail}

A compound computation consists of a computation followed by an optional list of computations separated by semicolons.
The computations are executed from left to right;
the value of the compound computation is the value of the last computation.
So you must give an expression at that point in order to return a value, 
whereas it is pointless to specify an expression before.

\begin{rail} 
  Computation:
     VariableDeclaration |
     Assignment |
     ProcedureMethodCall |
     ProcedureCall |
     lbrace SequenceExpression rbrace
  ;
	Assignment:	AssignmentTarget '=' (SequenceExpression | Assignment); 
\end{rail}\ixnterm{Computation}

A variable declaration declares a local variable in the same way as in the sequences.
An assignment assigns the value of a sequence expression to an assignment target.
It may be chained; such an assignment chain is executed from right to left, assigning the rightmost value to all the assignment targets given.
The expression used as computation -- denoted by and enclosed in braces -- will return a boolean value by comparing the return value of the expression to the default value of the corresponding type, returning false if equal, or true if unequal.
So just using a boolean variable as expression returns the value of the variable.
For returning an expression evaluation result value to a sequence you need two opening braces, one for entering the sequence computations, and the other for entering the sequence expressions.
The form of expressions and assignment targets will be specified below.

\begin{rail} 
	ProcedureMethodCall: ('(' Variables ')' '=')? (Variable | GraphElement '.' Attribute) \\ '.' MethodName '(' Arguments ')';
	ProcedureCall: ('(' Variables ')' '=')? ProcedureName '(' Arguments ')';
	Arguments: (SequenceExpression * ',');
	Varibales: (Variable + ',');
\end{rail}\label{recstmt}\indexmain{record}\indexmain{emit}

A method call executes a method on a variable, passing further arguments.
The method may be one of the predefined container methods, or a user-defined method (cf. \ref{sec:objectoriented}).

A procedure call executes a (built-in or user defined) procedure, passing further arguments.
In addition to the graph type based functions and procedures which will be explained in more detail further below,
\texttt{emit}, \texttt{record}, and \texttt{export} procedure calls can be given here: the emit procedure writes a double quoted string or the value of a variable to the emit target (stdout as default, or a file specified with the shell command \texttt{redirect emit}; in fact a sequence of the former may be given, which is to be prefered over string concatenation for performance reasons).
The record procedure writes a double quoted string or the value of a variable to the currently ongoing recordings (see \ref{recordnreplay}). This feature allows to mark states reached during the transformation process in order to replay only interesting parts of an recording. It is recommended to write only comment label lines, i.e. \verb/"#"/, some label, and \verb/"\n"/.
The export procedure exports the current graph to the path specified if called with one argument, or it exports the subgraph specified as first argument to the path specified as second argument.
It behaves like the export command from the GrShell, see \ref{outputcmds}.
Having it available in the sequences allows for programmed exporting, and exporting of parts of the graph, with the subgraph containment just computed.

Furthermore, a function \texttt{canonize(g:graph):string} is available,
which is intended to provide a canonical string representation for any graph, but currently does not work for all graphs.
The function currently uses the SMILES\cite{SMILES} method of producing an equitable partition of graph nodes, not a canonical order; while not offering full fledged graph canonization this algorithm is sufficient for many purposes. 
It allows to reduce graph comparisons to string comparisons, at the price of computing the Weininger algorithm for equitable partitions of nodes.

Moreover, the procedures from the built-in package \texttt{Debug} as explained in detail in \ref{secdebuggersubrule} may be called.
\texttt{Debug::add} when entering and \texttt{Debug::rem} when leaving a subrule computation of interest (always pairwise!),
\texttt{Debug::emit} to recored some subrule computation milestones, \texttt{Debug::halt} to halt the debugger, and \texttt{Debug::highlight} to highlight some graph elements in the debugger (halting it).

Besides those predefined procedures (and functions), you may call user procedures (and functions), defined in the rules file; cf. \ref{sub:compdef}.

Finally, an expression (without side effects) can be evaluated, this allows to return a (boolean) value from a computation.

\begin{rail}
  AssignmentTarget: 
    Variable (':' Type)? |
    'yield' Variable |
    GraphElement '.' Attribute |
    Variable '[' SequenceExpression ']' |
    GraphElement '.' Attribute '[' SequenceExpression ']' |
    GraphElement '.' 'visited' '[' SequenceExpression ']'
;
\end{rail}\ixnterm{AssignmentTarget}\ixkeyw{visited}\ixkeyw{yield}

Possible targets of assignments are the variables and def-variables to be yielded to, as in the simple assignments of the sequences. 
A \texttt{yield} assignment writes the rhs variable value to the lhs variable which must be declared as a  def-to-be-yielded-to variable (\texttt{def}-prefix) in the pattern containing the \texttt{exec} statement.
Yielding is only possible from compiled sequences, it always succeeds.
Further on, the attributes of graph elements may be written to, the values at given positions of array or deque or map variables may be written to, and the visited status of graph elements may be changed.

\begin{example}
The sequence computation \verb#{ x:int=42; y:N; (y)=proc(x); y.meth(x); y.a=x }# shows a variable declaration including an initialization (which falls out of scope at the closing brace), a variable declaration of node type without initialization, a call of a procedure with one input argument, assigning an output value, a call of a method of the node type, and the assignment of a graph element attribute.

The example \verb#{ x:array<int>=array<int>[]; x.add(42); x[0]=1; {x.size()>0} }# shows an array declaration and initialization, the adding of a value to the array, the indexed assignment to an array, and a terminal sequence expressions that makes the sequence computation succeed if the array is not empty (that's the case here) and fail otherwise. Sequence computations that don't end with a sequence expression always succeed.
\end{example}

%%%%%%%%%%%%%%%%%%%%%%%%%%%%%%%%%%%%%%%%%%%%%%%%%%%%%%%%%%%%%%%%%%%%%%%%%%%%%%%%%%%%%%%%%%%%%%%%
\section{Sequence Expression} \label{sec:seqexpr}

\begin{rail}
  SequenceExpression:  
    ConditionalSequenceExpression |
    BooleanSequenceExpression |
    RelationalSequenceExpression |
    ArtihemticSequenceExpression |
    PrimarySequenceExpression;
\end{rail}\ixnterm{SequenceExpression}

Sequence expressions are a subset of the expressions introduced in \ref{sub:expr}, containing the boolean and comparison operators, but only plus and minus as further operators.

\begin{rail}
  ConditionalSequenceExpression: 
    BooleanSequenceExpression '?' SequenceExpression ':' SequenceExpression;
\end{rail}\ixnterm{ConditionalSequenceExpression}

The conditional operator has lowest priority, if the condition evaluates to true the first expression is evaluated and returned, otherwise the second.

\begin{rail}
  BooleanSequenceExpression: 
    SequenceExpression (ampersand | xorhat | '|' | doubleampersand | '||') SequenceExpression |
    '!' SequenceExpression;
\end{rail}\ixnterm{BooleanSequenceExpression}

The boolean operators have the same semantics and same priority as in \ref{sub:expr}.

\begin{rail}
  RelationalSequenceExpression: 
    SequenceExpression ('==' | '!=' | '<' | '<='| '>' | '>=' | 'in' | titilde) SequenceExpression;
\end{rail}\ixnterm{RelationalSequenceExpression}

The equality operators work for every type and return whether the values to compare are equal or unequal.
The relational operators work as specified in \ref{sub:expr} for numerical types, in \ref{cha:container} for container types, and \ref{cha:graph} for graph type.

\begin{rail}
  ArithmeticSequenceExpression:
    SequenceExpression ('+' | '-' | '*' | '/' | percent) SequenceExpression;
\end{rail}\ixnterm{ArithmeticSequenceExpression}

The arithmetic operator plus is used to denote addition of numerical values or string concatenation,
the arithmetic operator minus is used to denote subtraction of numerical values.
Furthermore, you can multiply and divide numbers, or compute the remainder of a division.
Neither the arithmetic functions of package \texttt{Math}, nor the string methods are available in the sequence expressions.
Use the entities from the rule language for real computational work, you can reuse them easily, just call the functions from the sequence expressions, the procedures from the sequence statements, and the rules/tests from the sequences.

\begin{rail}
  PrimarySequenceExpression:
    BasicSequenceExpression |
    SpecialSequenceExpression;
\end{rail}\ixnterm{PrimarySequenceExpression}

The atoms of the sequence expressions are the basic and the special sequence expressions.

\begin{rail}
  BasicSequenceExpression:
    'def' '(' (Variable+',') ')' |
	  railat '(' NameString ')' |
 	  GraphElement '.' Attribute |
	  Variable | 
    Literal
  ;
\end{rail}\ixnterm{BasicSequenceExpression}\ixkeyw{def}

The basic sequence expressions are the foundational value sources.
A \texttt{def} term is successful iff all the variables are defined (not null).
The at operator allows to access a graph element by its \indexed{persistent name}.
The attribute access clause returns the attribute value of the given graph element.
The variable and literal basic expressions are the same as in the \emph{SimpleOrInteractiveExpression}, cf. \ref{sec:simplevarhandling};
this means esp. that a Variable may denote a graph global variable if prefixed with a double colon, here as well as in the AssignmentTarget.

\begin{rail}
  SpecialSequenceExpression:
    Variable '[' SequenceExpression ']' |
    GraphElement '.' Attribute '[' SequenceExpression ']' |
    GraphElement '.' 'visited' '[' SequenceExpression ']' |
    FunctionMethodCall |
    FunctionCall;
  ;
\end{rail}\ixnterm{SpecialSequenceExpression}\ixkeyw{visited}

The special sequence expressions are used for storage handling, for graph, subgraph and visited flag handling, for random value queries, for typeof queries, and for index access.

The storage oriented ones are used to access a storage or to call a method on a storage (note: here it is not possible to build method call chains). 
They were introduced in chapter \ref{cha:container}, and are summarized below in \ref{sec:storages}.

The graph and subgraph handling expressions allow to query the graph for its elements, the visited flags expressions allow to check whether a value is marked.
They were introduced in chapter \ref{cha:graph}, and are summarized below in \ref{sec:queryupdate}.

The random value function \texttt{random} behaves like the random function from the expressions, see \ref{sec:primexpr};
i.e. if noted down with an integer as argument it returns a random integer in between 0 and that upper bound, exclusive; if given without an argument it returns a random double in between 0.0 and 1.0, exclusive.

The \texttt{typeof} function returns the type \emph{as string}, for an arbitrary \GrG-object fed as input.
There is no type type supporting type comparisons as in the rule language existing, you are limited to string comparisons.

The index functions allow to fetch elements based on the name or the unique-id, or to retrieve the name or the unique-id.
Available are \texttt{nameof} to fetch the name of a node or edge or graph and \texttt{uniqueof} to fetch the unique id of a node or edge or graph.
The function \texttt{nodeByName} allows to retrieve a node by its name, \texttt{edgeByName()} does the same for an edge.
The function \texttt{nodeByUnique} allows to retrieve a node by its unique id, \texttt{edgeByUnique} does the same for an edge.
You find more on them in \ref{sec:performance}.

\begin{example}
The sequence expression \verb#{{ def(y) && y.meth(x)>=y.a || @("$1").a+1 != func(x) }}# checks whether the node typed variable \texttt{y} is defined, i.e. not null, and if so compares the return value of a method of that type with an attribute value. If the comparison succeeds, it defines the return value of the expression; in case the definedness check failed or the comparison yields false, is a graph element fetched from the graph by its persistent name \verb#$1#, and its attribute \texttt{a} plus 1 compared against the result of a function call. Then the result of this latter comparison defines the outcome of the expression.
\end{example}


%%%%%%%%%%%%%%%%%%%%%%%%%%%%%%%%%%%%%%%%%%%%%%%%%%%%%%%%%%%%%%%%%%%%%%%%%%%%%%%%%%%%%%%%%%%%%%%%
\section{Graph and Subgraph Based Queries and Updates}\label{sec:queryupdate}\label{sec:visited}

The graph and subgraph oriented parts of the sequence expressions are built from four groups,
the procedures for basic graph manipulation, the functions for querying the graph structure, the functions and procedures of the subgraph operations, and the visited flag query, assignment, and procedures.

The first group is built from basic graph manipulation operators, as defined in \ref{procstab} and described in \ref{sub:procedures}.
Elements may be added, removed, or retyped, and nodes may be merged or edges redirected.
Not available are numerical functions, they are only offered by the computations of the rules. 

\begin{example}
The sequence compuation \verb#{ (::x)=add(N); (::x)=retype(::x,M); rem(::x) }# adds a newly created node of type \texttt{N} to the graph (storing it in the global variable \verb#::x#), retypes it to \texttt{M}, and finally removes it again from the graph.
\end{example}

The second group is built from the operators querying primarily the connectedness of graph elements,
as defined in \ref{sub:functions}.
You may ask for one for the nodes or edges of a type.
You may query for the other for the source or target or opposite node of an edge.
Furthermore, you may query for adjacent nodes and incident edges,
maybe even transitively for the reachability.
Furthermore you may ask with a predicate whether nodes or edges are adjacent or incident to other nodes or edges, maybe even transitively for reachability.

\begin{example}
\verb#for{x:N in nodes(N); for{::y in outgoing(x); { ::z=target(::y); ::z.a = 42 } } }# 
is a sequences that sets the attribute \texttt{a} to 42 for all nodes that are adjacent as targets to a source node \texttt{x} of type \texttt{N}.
You will receive a runtime exception if the type of \verb#::z# does not possess an attribute \texttt{a}.

A more realistic example is to check whether two nodes returned by some rule applications are reachable from each other, carrying out a change only in this case:\\
\verb#(::x)=r() ;> (::y)=s() ;> if{ {{isReachable(::x,::y)}} ; doSomething(::x, ::y) }#
\end{example}

The third group is defined by functions and procedures that operate on (sub-)graphs, as defined in \ref{sub:functions} and \ref{sub:procedures}.
They are especially useful in state space enumeration, cf. \ref{sec:statespaceenum}.
To this end, parallelized graph isomorphy checking with the \texttt{equalsAny} function is especially of interest.
You may import, clone, or compute induced subgraphs.
You may export a subgraph or insert a subgraph into the hostgraph.

\begin{example}
\verb#( doSomething() ;> { File::export("graph"+i+".grs") } )*# is a sequence scheme for exporting a graph after each iteration step in a loop, gaining a series of snapshots on the hard drive.
In a later step, you may then conditionally add exported graphs to the host graph: 
\verb#if{cond; { ::g=File::import("graph"+n.a+".grs"); insert(::g) } }#
\end{example}

\begin{example}
When you model a state space with \texttt{Graph} representative nodes (\emph{not} \texttt{graph} standing for a real (sub)graph), which are pointing with \texttt{contains} edges to the nodes contained in their state (i.e. subgraph), and store additionally a replica of the subgraph in a \texttt{sub} attribute of the \texttt{Graph} node, so it is readily available for comparisons,
then the step of a state space enumeration with isomorphic state pruning is controlled with code like this:\\
\verb#<< modifiyCurrent(gr) ;; {adj=adjacent(gr, contains); sub=inducedSubgraph(adj)}#\\
Inside the backtracking double angles, a new state is computed as first step by modifying the currently focused state received as input \texttt{gr:Graph} from the previous step. The modified subgraph is extracted for comparison by computing the \texttt{inducedSubgraph} from the nodes \texttt{adjacent} via \texttt{contains}-edges to the \texttt{gr}-node.\\
\verb#;> for{others:Graph in nodes(Graph); {{sub!=others.sub}} } && #\\
The extracted subgraph is compared with all already enumerated subgraphs that can be accessed by their \texttt{Graph} representative node. Only if none is isomorphic to it, do we continue with making the state persistent.\\
\verb#/ {(ngr)=insertInduced(adj, gr)} && link(gr,ngr) && {ngr.sub=sub} /#\\
During a backtracking pause, the modified subgraph is cloned and inserted flatly into the host graph again with \texttt{insertInduced}. A link is added from the old representative to this new representative, to reflect ancestry. Then the subgraph attribute of the new representative \texttt{ngr} is filled with the previously computed subgraph \texttt{sub}. Remark: the first \texttt{inducedSubgraph} above does not contain the representative node and thus is missing all containment edges, too. This \texttt{insertInduced} includes the representative node and thus the containment edges. Syntactical remark: \texttt{inducedSubgraph} is used in an assignment with a function call as RHS, whereas \texttt{insertInduced} is employed from a procedure call which requires parenthesis around the output arguments.\\
\verb#&& stateStep(ngr, level+1) >>#\\
Finally, we continue state space construction with the next step, modifying the just inserted subgraph.
After this step returns (with \texttt{false} as result), do the backtracking double angles roll back the modification -- keeping the changes written during the pause untouched -- and execute \texttt{modifyCurrent} on the next match available in \texttt{gr}.
\end{example}

The fourth group are the visited flags related operations,
as described in chapter \ref{sub:visitedaccess}.
Available is an expression for reading a visited flag, an assignment for writing a visited flag, and procedures for managing the visited flags as defined in \ref{procstab}.

\begin{example}
Because of the need to allocate and deallocate them, the visited flags are typically used with code like this:
\verb#flag:int ;> {(flag)=valloc()} ;> r(flag) ;> {vfree(flag)}#\\
In addition, they may be read in the sequence expressions, and written in the sequence computations:
\verb#if{ {{!n.visited[flag]}} ; { n.visited[flag] = true } }#
\end{example}

In the sequences only the sequence expressions are available to compute the parameters for the functions and procedures, compared to the full-fledged expressions of the computations language.


%%%%%%%%%%%%%%%%%%%%%%%%%%%%%%%%%%%%%%%%%%%%%%%%%%%%%%%%%%%%%%%%%%%%%%%%%%%%%%%%%%%%%%%%%%%%%%%%
\section{Storage Handling in the Sequences}\label{sec:storages}\indexmain{storage}
Storages are variables of container (set/map/array/deque) type (cf. \ref{sec:builtingenerictypes}) storing nodes or edges.
They are primarily used in the sequences, from where they are handed in to the rules via \texttt{ref} parameters (but additionally container attributes in graph elements may be used as storages,
esp. for doing data flow analyses, cf. \ref{subsub:flow}).
They allow to decouple processing phases: the first run collects all graph elements relevant for the second run which consists of a sequence executed for each graph element in the container.
The splitting of transformations into passes mediated by container valued global variables allows for subgraph copying without model pollution, cf. \ref{subsub:copystructure}; please have a look at \ref{sub:mergesplit}, \ref{subsub:copystructure} and \ref{subsub:flow} regarding a discussion on when to use which transformation combinators and for storage examples.
They were already defined and described in \ref{cha:container}.
Here we only give some refinements and explanations of the semantics.

The methods \texttt{add,rem,clear} are available for all storages and allow to add elements to the container, remove elements from the container, or clear the container.
Their return value is the changed container, thus they allow to chain method calls on the container.

The methods \texttt{size,empty,peek} in contrast return the size of the container, whether the container is empty, or a certain element from the container and thus can't be chained.
The indexed query \texttt{v=m[k]} is available on map and array and deque types and returns the element at the specified index,
the indexed assignment \texttt{a[i]=v} overwrites the element at the specified index.
Further available are the sequence expression operator \texttt{in} for membership query.

The infix operators, the methods specific to a certain container type (i.e. not available for all container types), and the change assignments in contrast are \emph{not} available in the sequence computations, they are \emph{only} supported by the computations in the rule language .

You may iterate in the sequences with a for loop over the elements contained in a storage.

\begin{rail}
  RewriteFactor:
    'for' lbrace (Var 'in' SetVar | Var '->' Var 'in' MapVar | Var 'in' ArrayVar | Var '->' Var 'in' ArrayVar | Var 'in' DequeVar | Var '->' Var 'in' DequeVar) ';' RewriteSequence rbrace
    ;
\end{rail}\ixkeyw{in}\ixkeyw{for}\ixnterm{RewriteFactor}\label{forstorage}

The \texttt{for} command iterates over all elements in the set or array or deque, or all key-value pairs in the map or array or deque, and executes for each element / key-value pair the nested graph rewrite sequence; it completes successfully iff all sequences were executed successfully (an empty container causes immediate successful completion); the key in the key-value pair iteration of an array or deque is the integer typed index. (See \ref{forgraphelem} for another version of the \texttt{for} command.)

\begin{example}
The following XGRS is a typical storage usage.
First an empty set \texttt{x} is created, which gets populated by an rule \texttt{t} executed iteratedly, returning a node which is written to the set.
Then another rule is executed iterated for every member of the set doing the main work, and finally the set gets cleared to prevent memory leaks or later mistakes.
If the graph should stay untouched during set filling you may need \texttt{visited} flags to prevent endless looping.
\verb#x=set<Node>{} ;> ( (v)=t() && {x.add(v)} )+ && for{v in x; r(v)} <; {x.clear()}#
You could hand in the storage to the rule, and \texttt{add} there to the set, this would allow to shorten the sequence to:\\
\verb#x=set<Node>{} ;> ( t(x) )+ && for{v in x; r(v)} <; {x.clear()}#\\
The for loop could be replaced by employing the storage access in the rule construct, cf. \ref{sub:storageaccess}; this would be especially beneficial if the rule \texttt{r} inside the for loop would have to change the storage \texttt{x}, which would corrupt the iteration/enumeration variable.
\end{example}

\begin{warning}
The container over which the for loop iterates must stay untouched during iteration.
\end{warning}

%\begin{example}
%\begin{grgen}
%::x=set<NodeTypeA>{}
%y:map<Node,Edge> = map<Node,Edge>{u->v}
%\end{grgen}
%The first line declares or references a global variable \texttt{x} (without static type) and assigns the newly created, empty set of type \texttt{set<NodeTypeA>} to it as value.
%The second line declares a variable \texttt{y} of type \texttt{map<Node,Edge>} and assigns the newly created map of the same type containing a key value pair build from u mapping to v, where u is assumed to be a variable of node type and v to be a variable of edge type.
%\end{example}

%%%%%%%%%%%%%%%%%%%%%%%%%%%%%%%%%%%%%%%%%%%%%%%%%%%%%%%%%%%%%%%%%%%%%%%%%%%%%%%%%%%%%%%%%%%%%%%%
\section{Quick Reference Table}

Table~\ref{comptab} lists most of the operations of the graph rewrite computations at a glance.

 %\makeatletter
\begin{table}[htbp]
\begin{minipage}{\linewidth} \renewcommand{\footnoterule}{} 
\begin{tabularx}{\linewidth}{|lX|}
\hline
\texttt{c;d}	& Computes c then d; the value of the computation is d\\
\hline
\texttt{\{e\}} & An unspecified sequence expression executed for its result, defining the success of the computation.\\
\hline
\texttt{t=e}	& Simple assignment of an expression value to an assignment target\\
%\texttt{t=e=f}	& Chained assignment \\
\hline
\texttt{e ? f : g}	& Returns f if e evaluates to true, otherwise g \\
\texttt{e op f}	& For \texttt{op} being one of the boolean operators \texttt{||,|,\&,\&\&,\^\ } \\
\texttt{e op f}	& For \texttt{op} being one of comparison operators \texttt{==,!=,<,<=,>,>=,in} \\
\texttt{e op f}	& For \texttt{op} being one of arithmetic operators \texttt{+,-,*,/,\%} \\
\texttt{e + f}	& For string concatenation. \\
\hline
\texttt{v} & Variable. Assignment target or expression.\\
\texttt{v.name} & Attribute of graph element. Assignment target or expression.\\
\texttt{@(name)} & Return graph element of given name.\\
\texttt{def(\emph{Parameters})} & Check if all the variables are defined.\\
\texttt{random(upperBound)} & Returns random number from [0;upper bound[, if upper bound is missing from [0.0;1.0[.\\
\texttt{typeof(v)} & Returns the name of the type of the entity handed in.\\
\hline
\texttt{u=set<Node>\{\}}	& Example for container constructor, creates storage set and assigns to \texttt{u}.\\
\texttt{u[e]}	& Target value of \texttt{e} in \texttt{u}. Fails if \texttt{!(e in u)}. Assignment target or expression.\\
\hline
\texttt{f(...)}	& Calls one of the functions for graph querying defined in \ref{funcstab} and explained in \ref{cha:graph}. Or calls a user defined function. The numerical functions are \emph{not} available.\\
\hline
\texttt{(...)=p(...)}	& Calls one of the procedures for graph manipulation defined in \ref{procstab} and explained in \ref{cha:graph}. Or calls a user defined procedure.\\
\hline
\texttt{v.fm(...)}	& Calls one of the function methods \texttt{size,empty,peek} for container querying defined in \ref{funcmethstab} and explained in \ref{cha:container}. Or calls a user defined function. The string function methods and other container function methods as well as infix operators are \emph{not} available.\\
\hline
\texttt{(...)=v.pm(...)}	& Calls one of the procedure methods \texttt{add,rem,clear} for container manipulation defined in \ref{procmethstab} and explained in \ref{cha:container}. Or calls a user defined procedure. The change assignments are \emph{not} available.\\
\hline
\end{tabularx}
\end{minipage}\\
\\ 
{\small Let \texttt{c} and \texttt{d} be computations, \texttt{t} be an assignment target, \texttt{e}, \texttt{f}, \texttt{g} be expressions, \texttt{u}, \texttt{v}, \texttt{w} be variable identifiers }
\caption{Sequence computations at a glance}
\label{comptab}
\end{table}
%\makeatother
 
% todo: beispiele im text bringen
