\chapter{Types and Expressions}
\label{cha:typeexpr}
%In the following sections \emph{Ident} refers to an identifier of the graph model language (see Section~\ref{modelbb}) or the rule set language (see Section~\ref{rulebb}).
%\emph{TypeIdent} is an identifier of a node type or an edge type, \emph{NodeOrEdge} is an identifier of a node or an edge.


%%%%%%%%%%%%%%%%%%%%%%%%%%%%%%%%%%%%%%%%%%%%%%%%%%%%%%%%%%%%%%%%%%%%%%%%%%%%%%%%%%%%%%%%%%%%%%%%
\section{Built-In Types}
\label{sec:builtintypes}
Besides user-defined node types, edge types, and enumeration types (as introduced in Chapter~\ref{chapmodellang}), \GrG\ supports the built-in \indexed{primitive types}\indexmainsee{built-in types}{primitive types} in Table~\ref{builtintypes} and the built-in \indexed{generic types}\indexmainsee{built-in generic types}{generic types} in Table~\ref{builtingenerictypes}.
The exact type format is \indexed{backend} specific.
The \indexed{LGSPBackend} maps the \GrG\ primitive types to the corresponding C\# primitive types, and the generic types to generic C\#-Dictionaries (i.e. hashmaps) or C\#-Lists (misnamed dynamic arrays) of their corresponding primitive types, with \texttt{de.unika.ipd.grGen.libGr.SetValueType} as target type for sets, only used with the value \texttt{null}.

\begin{table}[htbp]
\begin{tabularx}{\linewidth}{|l|X|}
	\hline
	\texttt{\indexed{boolean}} & Covers the values \texttt{true} and \texttt{false} \\
  \texttt{\indexed{byte, short, int, long}} & A signed integer, with 8 bits, with 16 bits, with 32 bits, with 64 bits \\
	\texttt{\indexed{float}}, \texttt{\indexed{double}} & A floating-point number, with single precision, with double precision\\
	\texttt{\indexed{string}} & A character sequence of arbitrary length\\
	\texttt{\indexed{object}} & Contains a .NET object\\
	\hline
\end{tabularx}
\caption{\GrG\ built-in primitive types}
\label{builtintypes}
\end{table}

\begin{table}[htbp]
\begin{tabularx}{\linewidth}{|l|X|}
	\hline
	\texttt{\indexed{set}<T>} & A (mathematical) set of type \texttt{T}, where \texttt{T} may be an enumeration type or one of the primitive types from above; it may even be a node or edge type, then we speak of storages. \\
	\texttt{\indexed{map}<S,T>} & A (mathematical) map from type \texttt{S} to type \texttt{T}, where \texttt{S} and \texttt{T} may be enumeration types or one of the primitive types from above; it may even be a node or edge type, then we speak of storages. \\
	\texttt{\indexed{array}<T>} & An array of type \texttt{T}, where \texttt{T} may be an enumeration type or one of the primitive types from above; it may even be a node or edge type, then we speak of storages. Shares some similarities with \texttt{map<int,T>}. \\
	\hline
\end{tabularx}
\caption{\GrG\ built-in generic types}
\label{builtingenerictypes}
\end{table}

\begin{table}[htbp]
  \centering
  \begin{tabular}[c]{|c|ccccccc|} \hline
    \backslashbox{to}{from} & \texttt{enum} & \texttt{boolean} & \texttt{int} & \texttt{double} & \texttt{string} & \texttt{object} & \\ \hline
    \texttt{enum} & $=$/--- & & & & & &\\
    \texttt{boolean} & & $=$ & & & & &\\
    \texttt{int} & implicit & & $=$ & \texttt{(int)} & & & \\
    \texttt{double} &  implicit & & implicit & $=$ & & & \\
    \texttt{string} & implicit & implicit & implicit &  implicit & $=$ & implicit & \\
    \texttt{object} & &  & & & & $=$ &\\\hline
  \end{tabular}
  \caption{\GrG\ type casts}
  \label{tabcasts}
\end{table}

Table~\ref{tabcasts} lists \GrG's implicit \indexed{type cast}s and the allowed explicit type casts.
Of course you are free to express an implicit type cast by an explicit type cast as well as ``cast'' a type to itself.
The \texttt{int} is the default integer type, in the table it stands for all the integer types.
The \texttt{double} is the default floating point type, in the table it stands for all the floating points types.
The integer types are implicitely casted upwards from smaller to larger types (\texttt{byte < short < int < long}), whereas a downcast requires an explicit cast.
The \texttt{byte} and \texttt{short} types are not used in computations, they are casted up to \texttt{int} (or \texttt{long} if required by the context.)
The floating point types are implicitely casted upwards (\texttt{float < double}), and require an explicit cast downwards, too.
As specified by the table, integer numbers are automatically casted to floating point numbers, and castable with an explicit cast vice versa.
According to the table neither implicit nor explicit casts from {\tt int} to any \indexed{enum type} are allowed.
This is because the range of an enum type is very sparse in general.
For the same reason implicit and explicit casts between enum types are also forbidden.
Thus, enum values can only be assigned to attributes having the same enum type.
A cast of an enum value to a string value will return the declared name of the enum value.
A cast of an object value to a string value will return ``null'' or it will call the \texttt{toString()} method of the .NET object.
Everything is implicitely casted to \texttt{string} to enable concise text output without the need for boilerplate casting.
Be careful with assignments of objects: \GrG\ does not know your .NET type hierarchy and therefore it cannot check two objects for type compatibility.
Objects of type object are not very useful for \GrG\ processing and the im/exporters can't handle them,
but they can be used on the API level.

\begin{example}
  \begin{itemize}
    \item Allowed:\\
	  \texttt{x.myfloat = x.myint; x.mydouble = (float) x.myint;\\ x.mystring = (string) x.mybool;}
    \item Forbidden:\\
      \texttt{x.myfloat = x.mydouble;} and \texttt{x.myint = (int) x.mybool;}\\
      \texttt{MyEnum1 = (MyEnum1Type) int;} and \texttt{MyEnum2 = (MyEnum2Type) MyEnum1;}
  where {\tt myenum1} and {\tt myenum2} are different enum types.

  \end{itemize}
\end{example}

\begin{note}
	Unlike an {\tt eval} part (which must not contain assignments to node or edge attributes) the declaration of an enum type can contain assignments of {\tt int} values to \indexed{enum item}s (see Section~\ref{typedecl}).
	The reason is, that the range of an enum type is just defined in that context.
\end{note}


%%%%%%%%%%%%%%%%%%%%%%%%%%%%%%%%%%%%%%%%%%%%%%%%%%%%%%%%%%%%%%%%%%%%%%%%%%%%%%%%%%%%%%%%%%%%%%%%
\section{Expressions}\indexmain{expression}\label{sub:expr}

\GrG~supports numerous operations on the entities of the types introduced above, which are organized into left associative expressions.
In the following they will be explained with their semantics and relative priorities one type after another in the order of the rail diagram below.

\begin{rail}
  Expression: BoolExpr | RelationalExpr | IntExpr | FloatExpr | StringExpr | SetExpr | MapExpr | ArrayExpr | TypeExpr | PrimaryExpr;
\end{rail}\ixnterm{Expression}


%%%%%%%%%%%%%%%%%%%%%%%%%%%%%%%%%%%%%%%%%%%%%%%%%%%%%%%%%%%%%%%%%%%%%%%%%%%%%%%%%%%%%%%%%%%%%%%%
\section{Boolean Expressions}

The boolean expressions combine boolean values with logical operations.
They bind weaker than the relational expressions which bind weaker than the other expressions.

\begin{rail}
  BoolExpr: ((() | '!') PrimaryExpr) | (BoolExpr '?' BoolExpr ':' BoolExpr) | (BoolExpr BinBoolOperator BoolExpr) | RelationalExpr;
\end{rail}\ixnterm{BoolExpr}
The unary \texttt{!}\ operator negates a Boolean.
The binary \emph{BinBoolOperator} is one of the operators in Table~\ref{tabboolops}.
\begin{table}[htbp]
  \centering
  %\begin{tabularx}{0.45\linewidth}{|ll|} \hline
  \begin{tabular}[c]{|lp{0.6\linewidth}|} \hline
    \begin{tabular}[c]{l} \texttt{\^} \end{tabular} & \begin{tabular}[c]{l} Logical XOR. True, iff either the first or the second \\ Boolean expression is true. \end{tabular} \\ \hline
    \begin{tabular}[c]{l} \texttt{\&\&} \\ \texttt{||} \end{tabular} & \begin{tabular}[c]{l} Logical AND and OR. Lazy evaluation. \end{tabular}\\ \hline
    \begin{tabular}[c]{l} \texttt{\&} \\ \texttt{|} \end{tabular} & \begin{tabular}[c]{l} Logical AND and OR. Strict evaluation. \end{tabular}\\ \hline
  \end{tabular}
  \caption{Binary Boolean operators, in ascending order of precedence}\indexmain{order of precedence}\indexmainsee{precedence}{order of precedence}
  \label{tabboolops}
\end{table}
The ternary \texttt{?}\ operator is a simple if-then-else: If the first \emph{BoolExpr} is evaluated to \texttt{true}, the operator returns the second \emph{BoolExpr}, otherwise it returns the third \emph{BoolExpr}.


%%%%%%%%%%%%%%%%%%%%%%%%%%%%%%%%%%%%%%%%%%%%%%%%%%%%%%%%%%%%%%%%%%%%%%%%%%%%%%%%%%%%%%%%%%%%%%%%
\section{Relational Expressions}

The relational expressions compare enitites of different kinds, mapping them to the type boolean.
The bind stronger than the boolean expressions but weaker than all the other non-boolean expressions.

\begin{rail}
 RelationalExpr: (Expression CompareOperator Expression)
\end{rail}\ixnterm{RelationalExpr}

The \emph{CompareOperator} is one of the following operators:
\[ \texttt{<} \;\;\;\;\; \texttt{<=} \;\;\;\;\; \texttt{==} \;\;\;\;\; \texttt{!=} \;\;\;\;\; \texttt{>=} \;\;\;\;\; \texttt{>} \]
Their semantics are type dependent.

For arithmetic expressions on \texttt{int} and \texttt{float} or \texttt{double} types
the semantics is given by Table~\ref{compandarithmetic} (by implicit casting they can also by used with all enum types).

\begin{table}[htbp]
  \centering
  \begin{tabularx}{\linewidth}{|l|X|} \hline
    \texttt{A == B} & True, iff $A$ is the same number as $B$. \\
    \texttt{A != B} & True, iff $A$ is a different number than $B$. \\
    \texttt{A <\ \ B} & True, iff $A$ is smaller than and not equal $B$. \\
    \texttt{A >\ \ B} & True, iff $A$ is greater than and not equal $B$. \\
    \texttt{A <= B} & True, iff $A$ is smaller than (or equal) $B$. \\
    \texttt{A >= B} & True, iff $A$ is greater than (or equal) $B$. \\ \hline
  \end{tabularx}
  \caption{Compare operators on arithmetic expressions}
  \label{compandarithmetic}
\end{table}

\texttt{String} types, \texttt{boolean} types, and \texttt{object} types support only the \texttt{==} and the \texttt{!=} operators;
for strings they denote whether the strings are the same or not,
on boolean values they denote equivalence and antivalence,
and on object types the tell whether the references are the same, thus the objects identical.

For set/map/array expressions, table~\ref{compandsetmap} describes the semantics of the compare operators.
Some set A is a subset of B iff all elements in A are contained in B, too.
A map A is a submap of B iff all key-value pairs of A are contained in B, too. If they have a key in common but map to a different value, they count as not identical.
An array A is a subarray of B iff it is smaller or equal in size and the values at each common index are identical.
\begin{table}[htbp]
  \centering
  \begin{tabularx}{\linewidth}{|l|X|} \hline
    \texttt{A == B} & True, iff $A$ and $B$ are identical. \\
    \texttt{A != B} & True, iff $A$ and $B$ are not identical. \\
    \texttt{A <\ \ B} & True, iff $A$ is a subset/map/array of $B$, but $A$ and $B$ are not identical. \\
    \texttt{A >\ \ B} & True, iff $A$ is a superset/map/array of $B$, but $A$ and $B$ are not identical. \\
    \texttt{A <= B} & True, iff $A$ is a subset/map/array of $B$ or $A$ and $B$ are identical. \\
    \texttt{A >= B} & True, iff $A$ is a superset/map/array of $B$ or $A$ and $B$ are identical. \\ \hline
  \end{tabularx}
  \caption{Compare operators on set/map/array expressions}
  \label{compandsetmap}
\end{table}

For \indexed{type expression}s the semantics of compare operators are given by table~\ref{compandtypes},
the rule to remember is: types grow larger with extension/refinement. An example is given in \ref{typeexpressions}.
\begin{table}[htbp]
  \centering
  \begin{tabularx}{\linewidth}{|l|X|} \hline
    \texttt{A == B} & True, iff $A$ and $B$ are identical. Different types in a type hierarchy are \emph{not} identical. \\
    \texttt{A != B} & True, iff $A$ and $B$ are not identical. \\
    \texttt{A <\ \ B} & True, iff $A$ is a supertype of $B$, but $A$ and $B$ are not identical. \\
    \texttt{A >\ \ B} & True, iff $A$ is a subtype of $B$, but $A$ and $B$ are not identical. \\
    \texttt{A <= B} & True, iff $A$ is a supertype of $B$ or $A$ and $B$ are identical. \\
    \texttt{A >= B} & True, iff $A$ is a subtype of $B$ or $A$ and $B$ are identical. \\ \hline
  \end{tabularx}
  \caption{Compare operators on type expressions}
  \label{compandtypes}
\end{table}
\begin{note}
  \texttt{A < B} corresponds to the direction of the arrow in an \indexed{UML class diagram}.
\end{note}
\begin{note}
  \texttt{Node} and \texttt{Edge} are the least specific, thus bottom elements $\bot$ of the type hierarchy,\\
  i.e. the following holds:
  \begin{itemize}
    \item $\forall n\in Types_{Node}: Node <= n$
    \item $\forall e\in Types_{Edge}: Edge <= e$
  \end{itemize}
\end{note}


%%%%%%%%%%%%%%%%%%%%%%%%%%%%%%%%%%%%%%%%%%%%%%%%%%%%%%%%%%%%%%%%%%%%%%%%%%%%%%%%%%%%%%%%%%%%%%%%
\section{Arithmetic and Bitwise Expressions}

The arithmetic and bitwise expressions combinde integer and floating point values with the arithmetic operations usually available in programming languages
and integer values with bitwise logical operations (interpreting integer values as bit-vectors).

\begin{rail}
  IntExpr: ((() | '+' | '-' | tilde) PrimaryExpr) | (BoolExpr '?' IntExpr ':' IntExpr) | (IntExpr BinIntOperator IntExpr);
\end{rail}\ixnterm{IntExpr}
The $\sim$ operator is the bitwise complement.
That means every bit of an integer value will be flipped.
The \texttt{?}\ operator is a simple if-then-else: If the \emph{BoolExpr} is evaluated to \texttt{true}, the operator returns the first \emph{IntExpr}, otherwise it returns the second \emph{IntExpr}.
The \emph{BinIntOperator} is one of the operators in Table~\ref{tabbinops}.
\begin{table}[htbp]
  \centering
  %\begin{tabularx}{0.45\linewidth}{|ll|} \hline
  \begin{tabular}[c]{|lp{0.6\linewidth}|} \hline
    \begin{tabular}[c]{l} \texttt{\^} \\ \texttt{\&} \\ \texttt{|} \end{tabular} & \begin{tabular}[c]{l} Bitwise XOR, AND and OR \end{tabular} \\ \hline
    \begin{tabular}[c]{l} \texttt{\mbox{<}\mbox{<}} \\ \texttt{\mbox{>}\mbox{>}} \\ \texttt{\mbox{>}\mbox{>}\mbox{>}} \end{tabular} & \begin{tabular}[c]{l} Bitwise shift left, bitwise shift right and \\ bitwise shift right prepending zero bits (unsigned mode)\end{tabular}\\ \hline
    \begin{tabular}[c]{l} \texttt{+} \\ \texttt{-} \end{tabular} & \begin{tabular}[c]{l} Addition and subtraction \end{tabular}\\ \hline
    \begin{tabular}[c]{l} \texttt{*} \\ \texttt{/} \\ \texttt{\%} \end{tabular} & \begin{tabular}[c]{l}Multiplication, integer division, and modulo \end{tabular} \\ \hline
  \end{tabular}
  \caption{Binary integer operators, in ascending order of precedence}\indexmain{order of precedence}
  \label{tabbinops}
\end{table}

If one operand is \texttt{long} the operation is carried out with 64 Bits, otherwise the operation is carried out with 32 Bits, i.e. \texttt{int}-sized --- even if all the operands are of type \texttt{byte} or \texttt{short}.

\begin{rail}
  FloatExpr: ((() | '+' | '-') PrimaryExpr) | (BoolExpr '?' FloatExpr ':' FloatExpr) | (FloatExpr BinFloatOperator FloatExpr);
\end{rail}\ixnterm{FloatExpr}
The \texttt{?}\ operator is a simple if-then-else: If the \emph{BoolExpr} is evaluated to \texttt{true}, the operator returns the first \emph{FloatExpr}, otherwise it returns the second \emph{FloatExpr}.
The \emph{BinFloatOperator} is one of the operators in Table~\ref{tabfloatbinops}.
\begin{table}[htbp]
  \centering
  %\begin{tabularx}{0.45\linewidth}{|ll|} \hline
  \begin{tabular}[c]{|ll|} \hline
    \begin{tabular}[c]{l} \texttt{+} \\ \texttt{-} \end{tabular} & \begin{tabular}[c]{l} Addition and subtraction \end{tabular}\\ \hline
    \begin{tabular}[c]{l} \texttt{*} \\ \texttt{/} \\ \texttt{\%} \end{tabular} & \begin{tabular}[c]{l}Multiplication, division and modulo \end{tabular} \\ \hline
  \end{tabular}
  \caption{Binary float operators, in ascending order of precedence}\indexmain{order of precedence}
  \label{tabfloatbinops}
\end{table}
\begin{note}
The \texttt{\%} operator on float values works analogous to the integer modulo operator.
For instance \texttt{4.5 \% 2.3 == 2.2}.
\end{note}

If one operand is \texttt{double} the operation is carried out with 64 Bits, otherwise the operation is carried out with 32 Bits, i.e. \texttt{float}-sized.


%%%%%%%%%%%%%%%%%%%%%%%%%%%%%%%%%%%%%%%%%%%%%%%%%%%%%%%%%%%%%%%%%%%%%%%%%%%%%%%%%%%%%%%%%%%%%%%%
\section{String Expressions}

String expressions combine string values by string operations, with integer numbers used as helpers to denote positions in the strings (and giving the result of length counting).

\begin{rail}
  StringExpr: PrimaryExpr (MethodSelector)? | StringExpr '+' StringExpr;
\end{rail}\ixnterm{StringExpr}
The operator \texttt{+} concatenates two strings.
There are several operations on strings available in method call notation (MethodSelector), these are

\begin{description}
\item[\texttt{.length()}] returns length of string, as \texttt{int}
\item[\texttt{.indexOf(strToSearchFor)}] returns first position \texttt{strToSearchFor:string} appears at, as \texttt{int}, or -1 if not found
\item[\texttt{.lastIndexOf(strToSearchFor)}] returns last position \texttt{strToSearchFor:string} appears at, as \texttt{int}, or -1 if not found
\item[\texttt{.substring(startIndex, length)}] returns substring of given \texttt{length:int} from \texttt{startIndex:int} on
\item[\texttt{.replace(startIndex, length, replaceStr)}] returns string with substring from \texttt{startIndex:int} on of given \texttt{length:int} replaced by \texttt{replaceStr:int}
\end{description}

\begin{example}
For \texttt{n.str == "foo bar foo"} the operations yield \\
\texttt{n.str.length()==11} \\
\texttt{n.str.indexOf("foo")==0} \\
\texttt{n.str.lastIndexOf("foo")==8} \\
\texttt{n.str.substring(4,3)=="bar"} \\
\texttt{n.str.replace(4,3,"foo")=="foo foo foo"} \\
\end{example}


%%%%%%%%%%%%%%%%%%%%%%%%%%%%%%%%%%%%%%%%%%%%%%%%%%%%%%%%%%%%%%%%%%%%%%%%%%%%%%%%%%%%%%%%%%%%%%%%
\section{Set Expression}\label{sec:setexpr}

Set expressions consist of the known mathematical set operations, plus some operations in method call notation.

\begin{rail}
  SetExpr: Expr 'in' SetExpr | SetExpr (backslash | ampersand | '|') SetExpr | PrimaryExpr (MethodSelector)?;
\end{rail}\ixnterm{SetExpr}\ixkeyw{in}

\begin{table}[htbp]
  \centering
  %\begin{tabularx}{0.45\linewidth}{|ll|} \hline
  \begin{tabular}[c]{|ll|} \hline
    \begin{tabular}[c]{l} \verb#|# \end{tabular} & \begin{tabular}[c]{l}Set union (contained in resulting set as soon as contained in one of the sets)\end{tabular}\\ \hline
    \begin{tabular}[c]{l} \verb#&# \end{tabular} & \begin{tabular}[c]{l}Set intersection (contained in resulting set only if contained in both of the sets)\end{tabular} \\ \hline
    \begin{tabular}[c]{l} \verb#\# \end{tabular} & \begin{tabular}[c]{l}Set difference (contained in resulting set iff contained in left but not right set)\end{tabular} \\ \hline
  \end{tabular}
  \caption{Binary set operators, in ascending order of precedence}\indexmain{order of precedence}
  \label{tabsetbinops}
\end{table}

The binary set operators require the left and right operands to be of identical type \verb#set<T>#.
The operator \texttt{x in s} denotes set membership $x \in s$, returning whether the set contains the given element, as \texttt{boolean}.
Furthermore there are two operations on sets available in method call notation (MethodSelector):

\begin{description}
\item[\texttt{.size()}] returns the number of elements in the set, as \texttt{int}
\item[\texttt{.peek(num)}] returns the element which comes at position \texttt{num:int} in the sequence of enumeration, as \texttt{T} for \verb#set<T>#; the higher the number, the longer retrieval takes
\end{description}

\begin{note}
To add a value to a set you may use set union with a single valued set constructor,
to remove a value from a set you may use set difference with a single valued set constructor (for set constructors cf. \ref{sec:primexpr}).
\begin{grgen}
s | { "foo" }
s \ { n.a }
\end{grgen}
Used in this way they get internally optimized to the imperative set addition \texttt{s.add(x)} and removal \texttt{s.rem(x)} methods available in the \texttt{eval} block and the XGRS.
\end{note}


%%%%%%%%%%%%%%%%%%%%%%%%%%%%%%%%%%%%%%%%%%%%%%%%%%%%%%%%%%%%%%%%%%%%%%%%%%%%%%%%%%%%%%%%%%%%%%%%
\section{Map Expression} \label{sec:mapexpr}

Map expressions consist of the known mathematical set operations extended to maps, and map value lookup, plus some operations in method call notation.

\begin{rail}
  MapExpr: Expr 'in' MapExpr | MapExpr '[' Expr ']' | MapExpr (backslash | ampersand | '|') MapExpr | PrimaryExpr (MethodSelector)?;
\end{rail}\ixnterm{MapExpr}\ixkeyw{in}

\begin{table}[htbp]
  \centering
  %\begin{tabularx}{0.45\linewidth}{|ll|} \hline
  \begin{tabular}[c]{|ll|} \hline
    \begin{tabular}[c]{l} \verb#|# \end{tabular} & \begin{tabular}[c]{l}Map union: returns new map with elements which are in at least one of the maps,\\ with the value of map2 taking precedence\end{tabular}\\ \hline
    \begin{tabular}[c]{l} \verb#&# \end{tabular} & \begin{tabular}[c]{l}Map intersection: returns new map with elements which are in both maps,\\ with the value of map1 taking precedence\end{tabular} \\ \hline
    \begin{tabular}[c]{l} \verb#\# \end{tabular} & \begin{tabular}[c]{l}Map difference: returns new map with elements from map1\\ without the elements with a key contained in map2\end{tabular}\\ \hline
  \end{tabular}
  \caption{Binary map operators, in ascending order of precedence}\indexmain{order of precedence}
  \label{tabmapbinops}
\end{table}

The binary map operators require the left and right operands to be of identical type \verb#map<S,T>#,
with one exception for map difference, this operator accepts for a left operand of type \verb#map<S,T># a right operand of type \verb#set<S>#, too.
The operator \texttt{x in m} denotes map domain membership $x \in dom(m)$, returning whether the domain of the map contains the given element, as \texttt{boolean}.
The operator \texttt{m[x]} denotes map lookup, i.e. it returns the value \texttt{y} which is stored in the map \texttt{m} for the value \texttt{x} (domain value \texttt{x} is mapped by the mapping \texttt{m} to range value \texttt{y}). The value \texttt{x} \emph{must} be in the map, i.e. \texttt{x in m} must hold.
There are several operations on maps available in method call notation (MethodSelector), these are:

\begin{description}
\item[\texttt{.size()}] returns the number of elements in the map, as \texttt{int}
\item[\texttt{.domain()}] returns the set of elements in the domain of the map, as \verb#set<S># for \verb#map<S,T>#
\item[\texttt{.range()}] returns the set of elements in the range of the map, as \verb#set<T># for \verb#map<S,T>#
\item[\texttt{.peek(num)}] returns the key of the element which comes at position \texttt{num:int} in the sequence of enumeration, as \texttt{S} for \verb#map<S,T>#; the higher the number, the longer retrieval takes
\end{description}

\begin{note}
To add a key,value-pair to a map you may use map union with a single valued map constructor,
to remove a value from a map you may use map difference with a single valued set or map constructor (for map constructors cf. \ref{sec:primexpr}).
\begin{grgen}
m | { "foo" -> 42 }
m \ { n.a -> n.b } or m \ { n.a }
\end{grgen}
Used in this way they get internally optimized to the imperative map addition \texttt{s.add(key,value)} and removal \texttt{s.rem(key)} methods available in the \texttt{eval} block and the XGRS.
\end{note}


%%%%%%%%%%%%%%%%%%%%%%%%%%%%%%%%%%%%%%%%%%%%%%%%%%%%%%%%%%%%%%%%%%%%%%%%%%%%%%%%%%%%%%%%%%%%%%%%
\section{Array Expression} \label{sec:arrayexpr}

Array expressions consist of array membership checking, array value lookup, and array concatenation, plus some operations in method call notation.

\begin{rail}
  ArrayExpr: Expr 'in' ArrayExpr | ArrayExpr '[' Expr ']' | ArrayExpr '+' ArrayExpr | PrimaryExpr (MethodSelector)?;
\end{rail}\ixnterm{ArrayExpr}\ixkeyw{in}

The operator \texttt{x in a} denotes array value membership, returning whether the array contains the given element, as \texttt{boolean}.
The operator \texttt{a[x]} denotes array lookup, i.e. it returns the value \texttt{y} which is stored in the array \texttt{a} at the index \texttt{x}.
The index \texttt{x} \emph{must} be a valid array index.
The operator \texttt{x + y} denotes array concatenation, which requires the left and right operands to be of identical type \verb#array<T>#.
There are several operations on arrays available in method call notation (MethodSelector), these are:

\begin{description}
\item[\texttt{.size()}] returns the number of elements in the array, as \texttt{int}
\item[\texttt{.indexOf(valueToSearchFor)}] returns first position \texttt{valueToSearchFor:T} appears at, as \texttt{int}, or -1 if not found
\item[\texttt{.lastIndexOf(valueToSearchFor)}] returns last position \texttt{valueToSearchFor:T} appears at, as \texttt{int}, or -1 if not found
\item[\texttt{.subarray(startIndex, length)}] returns subarray of given \texttt{length:int} from \texttt{startIndex:int} on
\item[\texttt{.peek(num)}] returns the value stored in the array at position \texttt{num:int} in the sequence of enumeration, is equivalent to (and implemented by) \texttt{a[num])}; retrieval occurs in constant time.
\end{description}

\begin{note}
In contrast to sets and maps which can be handled well by expressions, arrays are meant to be modified by imperative \texttt{eval} block and XGRS statements like array addition \texttt{a.add(value)}/\texttt{a.add(value, index)} and removal \texttt{s.rem()}/\texttt{s.rem(index)}.
For declarative array construction please have a look at \ref{sec:primexpr}.
\end{note}


%%%%%%%%%%%%%%%%%%%%%%%%%%%%%%%%%%%%%%%%%%%%%%%%%%%%%%%%%%%%%%%%%%%%%%%%%%%%%%%%%%%%%%%%%%%%%%%%
\section{Type Expressions}\indexmain{type expression}
\label{typeexpressions}

\begin{rail}
  TypeExpr: TypeIdent | 'typeof' '(' NodeOrEdge ')' ;
\end{rail}\ixkeyw{typeof}\ixnterm{TypeExpr}
A type expression identifies a type (and---in terms of matching---also its subtypes).
A type expression is either a type identifier itself or the type of a graph element.
The type expression \texttt{typeof(x)} stands for the type of the host graph element \texttt{x} is actually bound to.

\begin{example}
\begin{tabularx}{\linewidth}{cX}
  \begin{tikzpicture}[baseline=(T.base)] \tt
    \begin{scope}[minimum size=0.5cm]
      \tikzstyle{every node}=[draw]
      \node (T)     at (1   ,4) {\texttt{T}};
      \node (T1)     at (1   ,3) {\texttt{T1}};
      \node (T2)     at (0   ,2) {\texttt{T2}};
      \node (T4)     at (0   ,1) {\texttt{T4}};
      \node (T3)     at (2   ,2) {\texttt{T3}};
    \end{scope}
    \draw[thick,-open triangle 45]  (T1) -> (T)  ;
    \draw[thick,-open triangle 45]  (T2) -> (T1)  ;
    \draw[thick,-open triangle 45]  (T3) -> (T1)  ;
    \draw[thick,-open triangle 45]  (T4) -> (T2)  ;
  \end{tikzpicture} &
  \parbox{\linewidth}{The expression \texttt{typeof(x)<=T2} applied to the type hierarchy on the left side yields \texttt{true} if \texttt{x} is a graph element of type \texttt{T} or \texttt{T1} or \texttt{T2}.
                      The expression \texttt{typeof(x)>T2} only yields \texttt{true} for \texttt{x} being a graph element of type \texttt{T4}. The expression \texttt{T1<T3} always yields \texttt{true}.}
\end{tabularx}
\end{example}


%%%%%%%%%%%%%%%%%%%%%%%%%%%%%%%%%%%%%%%%%%%%%%%%%%%%%%%%%%%%%%%%%%%%%%%%%%%%%%%%%%%%%%%%%%%%%%%%
\section{Primary expressions}\label{sec:primexpr}

After we've seen the all the ways to combine expressions, finally we'll have a look at the atoms the expressions are built of.

\begin{rail}
  PrimaryExpr: CastExpr
  		| MemberAccess
  		| FunctionCall
  		| VarIdent
  		| NodeOrEdge '.' 'visited' ('[' FlagNumber ']')?
  		| 'nameof' '(' NodeOrEdge? ')'
  		| 'random' '(' ( IntExpr )? ')'
  		| '(' Expression ')';
\end{rail}\ixnterm{PrimaryExpr}\ixkeyw{visited}\ixkeyw{nameof}\indexmainsee{variable}{expression variable}\indexmain{expression variable}\indexmain{variable}

The \texttt{visited} query returns the status of the \indexed{visited flag} of the given number for the given graph element as \texttt{boolean}.
If no flag number is given, the default number for the first visited flag of 0 is used.
The visited flags are written in the assignments of the eval statements (see section \ref{sec:replacemodify}).
Make sure to allocate \ref{allocvisitflag}/\ref{apiallocvisitflag} visited flags before you try to use them
(and deallocate them afterwards, as they are a sparse resource stored in some excess bits of the graph elements, or in some dictionary if the needed number of flags exceeds the number of available bits per graph element.)

The \texttt{nameof} query returns the name (persistent name, see example \ref{persistentex}) of the given graph element as \texttt{string};
graphs elements at the API level bear no name, the operator can only (sensibly) be used with Shell-graphs (\texttt{NamedGraph}; unless you decide to use the \texttt{NamedGraph} on API level, too).
If no graph element is given, the name of the graph is returned.

The \texttt{random} function returns a double random value in between 0.0 and 1.0 if called without an argument,
or, if an integer argument value is given, an integer random value in between 0 and the value given, excluding the value itself.

\begin{rail}
  CastExpr: '(' PrimitiveType ')' PrimaryExpr;
  MemberAccess: (NodeOrEdge '.' Ident);
  FunctionCall: FunctionName '(' (() | Expr + ',') ')';
\end{rail}\ixnterm{CastExpr}\ixnterm{MemberAccess}\ixnterm{FunctionCall}

The cast expression returns the original value casted to the new prefixed type.
The member access \texttt{n.a} returns the value of the attribute \texttt{a} of the graph element \texttt{n}.
A function call employs an external (attribute evaluation) function (cf. \ref{sub:extfct}) or one of the following built-in functions:

\begin{description}
\item[\texttt{min(.,.)}] returns the smaller of the two argument values, which must be of the same numeric type (i.e. both either \texttt{byte} or \texttt{short} or \texttt{int} or \texttt{long} or \texttt{float} or \texttt{double})
\item[\texttt{max(.,.)}] returns the greater of the two argument values, which must be of the same numeric type (i.e. both either \texttt{byte} or \texttt{short} or \texttt{int} or \texttt{long} or \texttt{float} or \texttt{double})
\item[\texttt{pow(.,.)}] returns the first argument value to the power of the second value; both must be of type \texttt{double}
\item[\texttt{incoming(.)}] returns a set of the edges which are incoming to the node given as argument value.
\item[\texttt{incoming(.,.)}] as above, but only edges of the edge type given as second argument are contained.
\item[\texttt{incoming(.,.,.)}] as above, but only if the edges originate from a node of the node type given as third argument.
\item[\texttt{outgoing(.)}] returns a set of the edges which are outgoing from the node given as argument value.
\item[\texttt{outgoing(.,.)}] as above, but only edges of the edge type given as second argument are contained.
\item[\texttt{outgoing(.,.,.)}] as above, but only if the edges lead to a node of the node type given as third argument.
\end{description}

\begin{rail}
  Literal: Constant | SetConstructor | MapConstructor | ArrayConstructor ;
  Constant: EnumLit | Number | HexNumber | FloatingNumber | QuotedText | BoolLit | NullLit;
\end{rail}\ixnterm{Constant}\ixnterm{Literal}

The Constants are:
\begin{description}
  \item[EnumLit] Is the value of an enum type, given in notation \texttt{EnumType '::' EnumValue}.
  \item[Number] Is a \texttt{byte} or \texttt{short} or \texttt{int} or \texttt{long} number in decimal notation without decimal point, postfixed by \texttt{y} or \texttt{Y} for \texttt{byte}, postfixed by \texttt{s} or \texttt{S} for \texttt{short}, postfixed by \texttt{l} or \texttt{L} for \texttt{long}, or of \texttt{int} type if not postfixed.
  \item[HexNumber] Is a \texttt{byte} or \texttt{short} or \texttt{int} or \texttt{long} number in hexadecimal notation starting with \texttt{0x}, the different types are distinguished by the suffix as for a decimal notation number.
  \item[FloatingNumber] Is a \texttt{float} or \texttt{double} number in decimal notation with decimal point, postfixed by \texttt{f} or \texttt{F} for \texttt{float}, maybe postfixed by \texttt{d} or \texttt{D} for \texttt{double} .
  \item[QuotedText] Is a string constant. It consists of a sequence of characters, enclosed by double quotes.
  \item[BoolLit] Is a constant of boolean type, i.e. one of the literals \texttt{true} or \texttt{false}.
  \item[NullLit] Is the one constant of object type, the literal \texttt{null}.
\end{description}

\begin{rail}
  SetConstructor: ('set' '<' Type '>')? \\ lbrace ( Expression*',' ) rbrace ;
  MapConstructor: ('map' '<' Type ',' Type '>')? \\ lbrace ( (Expression '->' Expression)*',' ) rbrace ;
  ArrayConstructor: ('array' '<' Type '>')? \\ '[' ( Expression*',' ) ']' ;
\end{rail}\ixnterm{SetConstructor}\ixnterm{MapConstructor}\ixnterm{ArrayConstructor}

The set/map/array constructors are constant if only primitive type literals, enum literals, or constant expressions are used; this is required for set/map/array initializations in the model.
They are non-constant if they contain nodes/edges/or member accesses, which is the common case if used in rules.
If the type of the set/map/array is given before the constructor, the elements given in the type constructor are casted to the specified member types if needed and possible.
Without the type prefix all elements given in the constructor must be of the same type.

\begin{example}
Some examples of literals:
\begin{grgen}
Apple::ToffeeApple // an enum literal
42y // an integer number in decimal notation of byte type
42s // an integer number in decimal notation of short type
42 // an integer number in decimal notation of int type
42L // an integer number in decimal notation of long type
0x7eadbeef // an integer number in hexadecimal notation of int type
0xdeadbeefL // an integer number in hexadecimal notation of long type
3.14159 // a double number
3.14159f // a float number
"ve rule and 0wn ze vorld" // a text literal
true // a bool literal
null // the object literal
{ "foo", "bar" } // a constant set<string> constructor
map<string,int>{ (n.strVal+m.strVal)->(m.intVal+n.intVal), intVal->strVal, "fool"->42 } // a non-constant map constructor with type prefix
[ 1,2,3 ] // a constant array<int> constructor
\end{grgen}
\end{example}


%%%%%%%%%%%%%%%%%%%%%%%%%%%%%%%%%%%%%%%%%%%%%%%%%%%%%%%%%%%%%%%%%%%%%%%%%%%%%%%%%%%%%%%%%%%%%%%%
\section{Operator Priorities}

The priorities of all available operators are shown in ascending order in the table below, the dots mark the positions of the operands, the commas separate the operators available on the respective priority level.

\begin{table}[htbp]
  \centering
  \begin{tabular}[c]{|ll|} \hline
    \begin{tabular}[c]{l} 01 \end{tabular} & \begin{tabular}[c]{l} \verb#. ? . : .# \end{tabular}\\ \hline
    \begin{tabular}[c]{l} 02 \end{tabular} & \begin{tabular}[c]{l} \verb#. || .# \end{tabular}\\ \hline
    \begin{tabular}[c]{l} 03 \end{tabular} & \begin{tabular}[c]{l} \verb#. && .# \end{tabular}\\ \hline
    \begin{tabular}[c]{l} 04 \end{tabular} & \begin{tabular}[c]{l} \verb#. | .# \end{tabular}\\ \hline
    \begin{tabular}[c]{l} 05 \end{tabular} & \begin{tabular}[c]{l} \verb#. ^ .# \end{tabular}\\ \hline
    \begin{tabular}[c]{l} 06 \end{tabular} & \begin{tabular}[c]{l} \verb#. & .# \end{tabular}\\ \hline
    \begin{tabular}[c]{l} 07 \end{tabular} & \begin{tabular}[c]{l} \verb#. \ .# \end{tabular}\\ \hline
    \begin{tabular}[c]{l} 08 \end{tabular} & \begin{tabular}[c]{l} \verb#. ==,!= .# \end{tabular}\\ \hline
    \begin{tabular}[c]{l} 09 \end{tabular} & \begin{tabular}[c]{l} \verb#. <,<=,>,>=,in .# \end{tabular}\\ \hline
    \begin{tabular}[c]{l} 10 \end{tabular} & \begin{tabular}[c]{l} \verb#. <<,>>,>>> .# \end{tabular}\\ \hline
    \begin{tabular}[c]{l} 11 \end{tabular} & \begin{tabular}[c]{l} \verb#. +,- .# \end{tabular}\\ \hline
    \begin{tabular}[c]{l} 12 \end{tabular} & \begin{tabular}[c]{l} \verb#. *,%,/ .# \end{tabular}\\ \hline
    \begin{tabular}[c]{l} 13 \end{tabular} & \begin{tabular}[c]{l} \verb#~,!,-,+ .# \end{tabular}\\ \hline
  \end{tabular}
  \caption{All operators, in ascending order of precedence}\indexmain{order of precedence}
  \label{tabopprios}
\end{table}

