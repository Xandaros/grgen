\chapter*{Foreword for Release 1.4}
First of all a word about the term ``graph rewriting''.
Some would rather say ``graph transformation''; some even think there is a difference between these two.
We don't see such differences and use graph rewriting for consistency.

The \textsc{GrGen} project started in spring 2003 with the diploma thesis of Sebastian Hack under supervision of Rubino Gei\ss.
At that time we needed a tool to find patterns in graph based intermediate representations used in compiler construction.
We imagined a tool that is fast, expressive, and easy to integrate into our compiler infrastructure.
So far Optimix\cite{assmann00graph} was the only tool that brought together the areas of compiler construction and graph rewriting.
However its approach is to feature many provable properties of the system per se, such as termination, confluence of derivations, and complete coverage of graphs.
This is achieved by restricting the expressiveness of the whole formalism below Turing-completeness.
Our tool \textsc{GrGen} in contrast should be Turing-complete.
Thus \GrG\ provides the user with strong expressiveness but leaves the task of proving such properties to the user.

To get a prototype quickly, we delegated the costly task of subgraph matching to a relational database system~\cite{Hac:03}.
Albeit the performance of this implementation could be improved substantially over the years, we believed that there was more to come.
Inspired by the PhD thesis of Heiko D\"orr~\cite{doerr} we reimplemented our tool to use search plan driven graph pattern matching of its own.
This matching algorithm evolved over time~\cite{adam,Bat:05:SA,Bat:05:DA,Bat:06,BKG:07} and has been ported from C to C\#~\cite{KG:07,Kro:07}.
In the year 2005 Varr\'o~\cite{gramot2005_adapt} independently proposed a similar search plan based approach.

Though we started four years ago to facilitate some compiler construction problems, in the meantime \GrG\ has grown into a general purpose tool for graph rewriting.\\[3ex]

We want to thank all co-workers and students that helped during the design and implementation of \GrG\ as well as the writing of this manual.
Especially we want to thank Dr.~Sebastian Hack, G.~Veit Batz, Michael Beck, Tom Gelhausen, Moritz Kroll, Dr.~Andreas Ludwig, and Dr.~Markus Noga.
Finally, all this would not happened without the productive atmosphere and the generous support that Prof.~Goos provides at his chair.\\[3ex]

We wish all readers of the manual---and especially all users of \GrG---a pleasant graph rewrite experience.
We hope you enjoy using \GrG\ as much as we enjoy developing it.\\[3ex]

Thank you for using \GrG.\\[6ex]

\noindent Karlsruhe in July 2007, Rubino Gei\ss~on behalf of the \GrG-Team

\pagebreak

%\medskip
%\smallskip

\chapter*{Foreword}

Since the preceding foreword was written a lot has happened: Dr. Rubino Gei\ss~finished his dissertation \cite{DissRuby} and left; Prof.\ Goos retired.
The succeeding Professor had no commitment to graph rewriting,
so \textsc{GrGen} switched from a university project developed by students and assistants
to an open source project (which is still hosted at the IPD, reachable from \url{www.grgen.net}).

The first steps of development from version 1.4 onwards, at that time still at Karlsruhe University, were the porting of \GrG\ from C to C\# \cite{Kro:07} by Moritz Kroll, which allowed for a faster pace of development,
and the addition of support for undirected edges and fine grain pattern conditions \cite{SABuchwald:2008} by Sebastian Buchwald.
Many thanks to them for their work towards the milestone version 2.5 \cite{ExpressiveConvenientFast:2010} and for their continued support. 

Another first step was the addition of alternative patterns and subpatterns by me in the context of my master's thesis, allowing for structural recursion \cite{Jak:08,StructuralRecursion}. 
While working on this thesis I fell in love with \GrG.
It did several things in just the most elegant way possible -- but it had weak spots regarding not-statically-fixed patterns as well as programmability and extensibility.
People in love do foolish things: after graduating I devoted much of my free time of the upcoming years to continue developing, in order to close the gaps.
I was dragged along by the feeling that graph rewriting is an underrated \emph{programming paradigm}, just lacking a general-purpose language and a development environment advanced enough to materialize that vision. (Besides programming language design is a highly interesting activity.)
With version 4 of \GrG\ this should have been accomplished.\\[2ex]

%In contrast to the prototypical implementation of an embedding of \GrG\ as a domain specific language into C\# \cite{DAMoritz} that is destroyed as soon as the internals of the hosting compiler change, did I follow the path of extending and generalizing the available languages \cite{ExpressiveConvenientFast:2010}.

%feedback from users and use cases:\\
We want to thank the organizers of GraBaTs\cite{grabats}/TTC, the annual meeting of the graph rewrite tool community,
which gave us the possibility to ruthlessly steal the best ideas of the competing tools.
Thanks to Berthold Hoffmann, and the discussions in ``French''.
And many thanks to several early users giving valuable feedback, a lot of bug reports, and even code (which is of course the best contribution you can give to an open source project), by name:
Bugra Derre, Paul Bedaride, Normen Müller, Pieter van Gorp and Nicholas Tung.
Many thanks to all the others that contributed since then.
If \emph{you} want to contribute or if you have a question, don't hesitate to contact the \GrG-Team
via email to \texttt{grgen} at the host given by \texttt{ipd.info.uni-karlsruhe.de}.\\[2ex]

The rule-and-pattern-based transformation of graph-based representations is a perfect fit for know\-ledge-intensive tasks, for tasks built on models where the relations play a central part, and for linguistic applications.
A much better fit than predicate logic or the lambda calculus typically misused for modelling them, or relational algebra that funnily fails when the relations become the important part, or tedious traditional programming.
Go and build a great application (e.g. an AI that enslaves humanity :) with \GrG\ for them.\\[2ex]

We wish you a pleasant graph rewriting experience.\\[2ex]

\noindent Karlsruhe in June 2014, Edgar Jakumeit on behalf of the \GrG-Team
