\chapter{Extensions}\indexmain{extensions}\label{chapextensions}

This chapter lists the ways you can customize \GrG without \GrG-means, the things which allow to interact with the external world outside of \GrG: attribute types, functions, annotations, and command line parameters.

%-----------------------------------------------------------------------------------------------
\section{External Attribute Types}\label{sub:extcls}
\begin{rail}
  ExternalClassDeclaration: 'class' IdentDecl (() | 'extends' (NodeType+',')) ';';
\end{rail}\ixnterm{ExternalClassDeclaration}\ixkeyw{class}\ixkeyw{extends}
Registers a new attribute type with \GrG. You may declare the base types of the type, but not give attributes. The attribute type must be implemented externally, see \ref{sub:extclsfctimpl}; for \GrG~the type is opaque, only external functions can do computations with it. You may extend \GrG~with external attribute types if the built-in attribute types (cf. \ref{sec:builtintypes}) are insufficient for your needs.

%-----------------------------------------------------------------------------------------------
\section{External Function Types}\label{sub:extfct}
\begin{rail}
  ExternalFunctionDeclaration: IdentDecl '(' ( () | (Type + ',') ) ')' ':' Type ';';
\end{rail}\ixnterm{ExternalFunctionDeclaration}
Registers an \indexed{external function} with \GrG~to be used in attribue computation.
An external function declaration specifies the expected input types and the output type. The function must be implemented externally, see \ref{sub:extclsfctimpl}.
An external function call (cf. \ref{sec:primexpr}) may receive and return values of the built-in (attribute) types as well as of the external attribute types; the real arguments on the call sites are type-checked against the declared signature following the subtyping hierarchy of the built-in as well as of the external attribute types.
You may extend \GrG~with external functions if the built-in attribute computation capabilities (cf. \ref{sub:expr}) are insufficient for your needs.

%-----------------------------------------------------------------------------------------------
\section{Shell Commands and Compiler Parameters}
\label{shell commands}

\noindent When executing the \GrG\ generator/compiler \texttt{GrGen.exe}, the follwing parameters are admissible:

\noindent \texttt{[mono] GrGen.exe } \texttt{[-keep [<dest-dir>]] [-debug]} \texttt{[-r <assembly-path>]}

The assembly \emph{assembly-path} is linked as reference to the compilation result with the \texttt{-r} parameter.

\begin{rail}
  'new' 'add' 'reference' Filename
\end{rail}\ixkeyw{new}\ixkeyw{add}\ixkeyw{reference}
Configures a reference to an external assembly \emph{Filename} to be linked into the generated assemblies, maps to the \texttt{-r} option of \texttt{grgen.exe} (cf. \ref{grgenoptions}).

\begin{rail}
  'new' 'set' ('keepdebug'|'lazynic') ('on'|'off')
\end{rail}\ixkeyw{new}\ixkeyw{set}\ixkeyw{keepdebug}\ixkeyw{lazynic}\ixkeyw{on}\ixkeyw{off}
Configures the compilation of the generated assemblies to keep the generated files and to add debug symbols,
or configures the generation of the matchers to execute negatives, independents, and conditions only at the end of matching (normally asap).
Maps to the \texttt{-keep} and the \texttt{-debug} options or to the \texttt{-lazynic} option of \texttt{grgen.exe} (cf. \ref{grgenoptions}).


%%%%%%%%%%%%%%%%%%%%%%%%%%%%%%%%%%%%%%%%%%%%%%%%%%%%%%%%%%%%%%%%%%%%%%%%%%%%%%%%%%%%%%%%%%%%%%%%
\section{Annotations}\indexmain{annotation}
\label{annotations}

Identifier \indexed{definition}s can be annotated by \indexedsee{pragma}{annotation}s. Annotations are key-value pairs.
\begin{rail}
  IdentDecl: Ident (() | '[' (Ident '=' Constant + ',') ']');
\end{rail}\ixnterm{IdentDecl}
Although you can use any key-value pairs between the brackets, only the identifier \indexed{prio} has an effect so far.
But you may use the annotations to transmit information from the specification files to API level where they can be enumerated.
\begin{table}[htbp]
\begin{tabularx}{\linewidth}{|lllX|} \hline
  \textbf{Key} & \textbf{Value Type} & \textbf{Applies to} & \textbf{Meaning} \\ \hline
  \texttt{prio} & int & node, edge & Changes the ranking of a graph element for \indexed{search plan}s. The default is \texttt{prio}=1000. Graph elements with high values are likely to appear prior to graph elements with low values in search plans.\\ \hline
\end{tabularx}
\caption{Annotations}
\label{tabannotations}
\end{table}
\begin{example}
We search the pattern \texttt{v:NodeTypeA -e:EdgeType-> w:NodeTypeB}. We have a host graph with about 100 nodes of \texttt{NodeTypeA}, 1,000 nodes of \texttt{NodeTypeB} and 10,000 edges of \texttt{EdgeType}. Furthermore we know that between each pair of \texttt{NodeTypeA} and \texttt{NodeTypeB} there exists at most one edge of \texttt{EdgeType}. \GrG\ can use this information to improve the initial search plan if we adjust the pattern like \texttt{v[prio=10000]:NodeTypeA -e[prio=5000]:EdgeType-> w:NodeTypeB}.
\end{example}

% todo: The maybeDeleted by Sebastian?
